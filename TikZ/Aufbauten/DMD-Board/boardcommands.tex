%%%%%%%%%%%%%%%%%%%%%%%%%%%%%%%%
% 
% 
%
%%%%%%%%%%%%%%%%%%%%%%%%%%%%%%%%


\newcommand{\aperture}[4]
{
	%#2 angle perpendicular to the surface
	%#3 width of object
	%#4 width of aperture
	%apertures do not influence the beam. thus we do not do any beam-related in this function
	\tikzmath{
	coordinate \M;
	\M = (#1);
	\alength = #3;
	\aper = #4;
	\angA = #2;
	};

	\path (\M) + (\angA+90:\aper) coordinate (H);
	\path (\M) + (\angA-90:\aper) coordinate (J);
	\path (\M) + (\angA+90:\alength/2) coordinate (K);
	\path (\M) + (\angA-90:\alength/2) coordinate (L);

	\draw[style=very thick](H) -- (K);
	\draw[style=very thick](J) -- (L);
}

\newcommand{\lens}[4]
{
	%#2 angle perpendicular to the surface and directed to the flat side
	%#3 diameter of lens
	%#4 beamwidth at lens surface
	
	\tikzmath{
	coordinate \M;
	\M = (#1);
	\ang = #2;
	\ldiameter = #3;
	\bwidth = #4;
	};	
	
	\path (\M) + (\ang+90:\ldiameter/2) coordinate (H);
	\path (\M) + (\ang+90:\bwidth/2) coordinate (BI1);
	\path (\M) + (\ang-90:\bwidth/2) coordinate (BI2);
	
	\draw[fill=blue!40] (H)  arc (\ang-210:\ang-150:\ldiameter) -- cycle;
	\fill[color = red, opacity = \op] (BO1) -- (BO2) -- (BI1) -- (BI2) -- cycle;
	
	\coordinate (BO1) at (BI2);
	\coordinate (BO2) at (BI1);
}

\newcommand{\DMD}[3]
{
	%(#1) center of DMD surface
	%(#2) next object
	%#3 beamwidth at DMD surface
	
	\tikzmath{
	coordinate \S, \Next;
	\S = (#1);
	\Next = (#2);
	\indx = \outx - \Sx;
	\indy = \outy - \Sy;
	if \indx < 0 then {
	\h = 180;
	} else {
	\h = 0;
	};
	\angIn = atan(\indy/\indx) + \h;	
	\outdx = \Nextx - \Sx;
	\outdy = \Nexty - \Sy;
	if \outdx < 0 then {
	\h2 = 180;
	} else {
	\h2 = 0;
	};
	\angOut = atan(\outdy/\outdx) + \h2;
	\ang = \angOut;
	\bwidth = #3/(cos(\angOut-\angIn));
	\w = \bwidth;
	\outx = \Sx;
	\outy = \Sy;
	};

	%Case
	\path (\S) + (\ang:1.3) coordinate (M);
	\path (M) + (\ang-90:3.6) coordinate (H);
	\path (H) + (\ang+180:4.5) coordinate (J);
	\path (J) + (\ang+90:6.5) coordinate (K);
	\path (K) + (\ang:4.5) coordinate (L);
	\draw[fill=black!20] (H) -- (J) -- (K) -- (L) -- cycle;
	
	%Surface
	\def\Slength{0.66};
	\path (\S) + (\ang+90:\Slength/2) coordinate (H);
	\path (\S) + (\ang+90:\bwidth/2) coordinate (BI1);
	\path (\S) + (\ang-90:\Slength/2) coordinate (J);
	\path (\S) + (\ang-90:\bwidth/2) coordinate (BI2);
	\draw[color = teal, style = ultra thick] (H) -- (J);
	
	\fill[color = red, opacity = \op] (BO1) -- (BO2) -- (BI1) -- (BI2) -- cycle;
	
	\coordinate (BO1) at (BI1);
	\coordinate (BO2) at (BI2);
}

\newcommand{\mirror}[4]
{
	%(#2) next object in beam	
	%#3 diameter of mirror
	%#4 beamwidth on mirror
	\def\mlength{#3}; 
  	
  	\tikzmath{
  	coordinate \M, \Next;
	\M = (#1);
	\Next = (#2);
	\indx = \outx - \Mx;
	\indy = \outy - \My;
	\outdx = \Nextx - \Mx;
	\outdy = \Nexty - \My;		
	if \indx < 0 then {
	\h = 180;
	} else {
	\h = 0;
	};
	if \indx != 0 then {
	\angIn = atan(\indy/\indx) + \h;
	} else {
	if \indy > 0 then {
	\angIn = 90;
	} else {
	\angIn = 270;
	};	
	};		
	if \outdx < 0 then {
	\h2 = 180;
	} else {
	\h2 = 0;
	};
	if \outdx != 0 then {
	\angOut = atan(\outdy/\outdx) + \h2;	
	} else {
	if \outdy > 0 then {
	\angOut = 90;
	} else {
	\angOut = 270;
	};
	};		
	\angM = (\angOut+\angIn)/2;
	\bwidth = #4/(cos(\angM-\angIn));
	\outx = \Mx;
	\outy = \My;
	};
  	
	\path (\M) + (\angM+90:\mlength/2) coordinate (H);
	\path (\M) + (\angM+90:\bwidth/2) coordinate (BI1);
	\path (H) + (\angM-90:\mlength) coordinate (J);
	\path (\M) + (\angM-90:\bwidth/2) coordinate (BI2);

	\draw[style=very thick](H) -- (J);
	\fill[color = red, opacity = \op] (BO1) -- (BO2) -- (BI1) -- (BI2) -- cycle;
	
	\coordinate (BO1) at (BI1);
	\coordinate (BO2) at (BI2);
}

\newcommand{\pickoff}[4]
{
	%(#2) next object in beam	
	%#3 diameter of mirror
	%#4 beamwidth on mirror
	\def\mlength{#3}; 
  	
  	\tikzmath{
  	coordinate \M, \Next;
	\M = (#1);
	\Next = (#2);
	\indx = \outx - \Mx;
	\indy = \outy - \My;
	\outdx = \Nextx - \Mx;
	\outdy = \Nexty - \My;		
	if \indx < 0 then {
	\h = 180;
	} else {
	\h = 0;
	};
	if \indx != 0 then {
	\angIn = atan(\indy/\indx) + \h;
	} else {
	if \indy > 0 then {
	\angIn = 90;
	} else {
	\angIn = 270;
	};	
	};		
	if \outdx < 0 then {
	\h2 = 180;
	} else {
	\h2 = 0;
	};
	if \outdx != 0 then {
	\angOut = atan(\outdy/\outdx) + \h2;	
	} else {
	if \outdy > 0 then {
	\angOut = 90;
	} else {
	\angOut = 270;
	};
	};		
	\angM = (\angOut+\angIn)/2;
	\bwidth = #4/(cos(\angM-\angIn));
	\outx = \Mx;
	\outy = \My;
	};
  	
	\path (\M) + (\angM+90:\mlength/2) coordinate (H);
	\path (\M) + (\angM+90:\bwidth/2) coordinate (BI1);
	\path (H) + (\angM-90:\mlength) coordinate (J);
	\path (\M) + (\angM-90:\bwidth/2) coordinate (BI2);

	\draw[color = blue!40, style=very thick](H) -- (J);
	\fill[color = red, opacity = \op] (BO1) -- (BO2) -- (BI1) -- (BI2) -- cycle;
	
	\coordinate (BO1) at (BI1);
	\coordinate (BO2) at (BI2);
	\coordinate (OB1) at (BO1);
	\coordinate (OB2) at (BO2);
	\tikzmath{
	\op = 0.4*\op;
	};
}

\newcommand{\avoidcrossedbeam}{
\coordinate (H) at (BO1);
\coordinate (BO1) at (BO2);
\coordinate (BO2) at (H);
}

\newcommand{\photodiode}[3]{
	%#2 angle
	%#3 point on original beam for correct angles
	\tikzmath{
	coordinate \M;
	\M = (#1);
	\ang = #2;
	\l = 5.7;
	\b = 3.5;
	};
	
	\draw[fill = black] (\M) circle (0.3);
	
	\path (\M) + (\ang+90:\l/2) coordinate (A);
	\path (A) + (\ang+180:\b) coordinate (B);
	\path (B) + (\ang-90:\l) coordinate (C);
	\path (C) + (\ang:\b) coordinate (D);
	
	\draw[fill= black!30] (A) -- (B) -- (C) -- (D) -- cycle;
	
	\fill[color = red, opacity = \op] (BO1) -- (BO2) -- (\M) -- cycle;
	
	\tikzmath{
	coordinate \H;
	\H = (#3);
	\outx = \Hx;
	\outy = \Hy;
	\op = 2*\op;
	};
	\coordinate (BO1) at (OB2);
	\coordinate (BO2) at (OB1);
}


\newcommand{\cube}[3]
{
	%#1 center of cube
	%#2 angle perpendicular to the fraction surface
	%#3 sidelength of cube
	% UNDER ONSTRUCTION: laserbeams in cube
	\coordinate (M) at (#1);
	\def\ang{#2};
	\def\sidelength{#3};

	\path (M) + (\ang+45:\sidelength/2) coordinate (H);
	\path (H) + (\ang-45:\sidelength/2) coordinate (A);
	\path (A) + (\ang-135:\sidelength) coordinate (B);
	\path (B) + (\ang+135:\sidelength) coordinate (C);
	\path (C) + (\ang+45:\sidelength) coordinate (D);
	\draw[fill=blue!40] (A) -- (B) -- (C) -- (D) -- cycle;
	\draw[] (B) -- (D);	
}

\newcommand{\plate}[3]
{
	%(#1) center of plate on incoming site
	%#2 angle perpendicular to the plate surface
	%#3 sidelength
	\coordinate (M) at (#1);
	\def\ang{#2};
	\def\sidelength{#3};
	\def\thickness{0.4};

	\path (M) + (\ang+90:\sidelength/2) coordinate (A);
	\path (A) + (\ang+180:\thickness) coordinate (B);
	\path (B) + (\ang-90:\sidelength) coordinate (C);
	\path (C) + (\ang:\thickness) coordinate (D);
	
	\draw[fill=blue!40] (A) -- (B) -- (C) -- (D) -- cycle;	
}

\newcommand{\test}[1]{
	%\node at (#1){Test};
	\tikzmath{
	coordinate \B;
	\B = (#1);
	};
	\node at (\B){\Bx};
}

\newcommand{\fibercollimator}[5]
{
	%#1 coordinates start of beam
	%#2 end of beam
	%#3 end of fiber
	%#4 alignment of fiber
	%#5 beamwidth at output

	\def\angF{#4};
	\def\bwidth{#5};
	\coordinate (F2) at (#3);	
		
	\tikzmath{
	coordinate \M, \Next;
	\M = (#1);
	\Next = (#2);	
	\x = \Nextx - \Mx;
	\y = \Nexty - \My;	
	if \x < 0 then {
	\h = 180;
	} else {
	\h = 0;
	};
	\angB = atan(\y/\x) + \h;
	\outx = \Mx;
	\outy = \My;
	};
		
	% Case
	\path (\M) + (\angB+180:0.15) coordinate (H1);
	\path (H1) + (\angB+90:2.1) coordinate (A2);
	\path (A2) + (\angB+180:2.1) coordinate (B2);
	\path (B2) + (\angB-90:4.2) coordinate (C2);
	\path (C2) + (\angB:2.1) coordinate (D2);
	\draw[fill=black] (A2) -- (B2) -- (C2) -- (D2) -- cycle;
	
	% Collimator
	\path (\M) + (\angB+90:0.6) coordinate (A1);
	\path (\M) + (\angB+90:\bwidth/2) coordinate (BO1);
	\path (\M) + (\angB-90:\bwidth/2) coordinate (BO2);
	\path (A1) + (\angB+180:2.4) coordinate (B1);
	\path (B1) + (\angB-90:0.6) coordinate (F1);
	\path (F1) + (\angB-90:0.6) coordinate (C1);
	\path (C1) + (\angB:2.4) coordinate (D1);
	\draw[fill=gray] (A1) -- (B1) -- (C1) -- (D1) -- cycle;
	
	% Fiber
	\draw[out=\angB+180, in=\angF, color = blue, style = ultra thick] (F1) to (F2);	
	
}

\newcommand{\camera}[3]
{
	%#1 coordinates of camera
	%#2 coordinates of last optic	
	%#3 beam width
		
	\tikzmath{
	coordinate \M, \Next;
	\M = (#1);
	\Next = (#2);	
	\bwidth = #3;
	\x = \Nextx - \Mx;
	\y = \Nexty - \My;	
	if \x < 0 then {
	\h = 180;
	} else {
	\h = 0;
	};
	\angB = atan(\y/\x) + \h;
	\outx = \Mx;
	\outy = \My;
	};
		
	% Tubus
	\path (\M) + (\angB+90:1.0) coordinate (A2);
	\path (A2) + (\angB+180:0.6) coordinate (B2);
	\path (B2) + (\angB-90:2.0) coordinate (C2);
	\path (C2) + (\angB:0.6) coordinate (D2);
	\draw[fill=black] (A2) -- (B2) -- (C2) -- (D2) -- cycle;
	
	% Case
	\path (\M) + (\angB+180:0.6) coordinate (H1);
	\path (H1) + (\angB+90:2.1) coordinate (A1);
	\path (A1) + (\angB+180:3.1) coordinate (B1);
	\path (B1) + (\angB-90:4.2) coordinate (C1);
	\path (C1) + (\angB:3.1) coordinate (D1);	
	\draw[fill=gray] (A1) -- (B1) -- (C1) -- (D1) -- cycle;
	
	% Beam
	\path (\M) + (\angB+90:\bwidth/2) coordinate (BI1);
	\path (\M) + (\angB-90:\bwidth/2) coordinate (BI2);
	
	\fill[color = red, opacity = \op] (BO1) -- (BO2) -- (BI1) -- (BI2) -- cycle;
}

\newcommand{\cameraD}[3]
{
	%#1 coordinates of camera
	%#2 coordinates of last optic	
	%#3 beam width
		
	\tikzmath{
	coordinate \M, \Next;
	\M = (#1);
	\Next = (#2);	
	\bwidth = #3;
	\x = \Nextx - \Mx;
	\y = \Nexty - \My;	
	if \x < 0 then {
	\h = 180;
	} else {
	\h = 0;
	};
	\angB = atan(\y/\x) + \h;
	\outx = \Mx;
	\outy = \My;
	};
		
	% Tubus
	\path (\M) + (\angB+90:1.6) coordinate (A2);
	\path (A2) + (\angB+180:2.1) coordinate (B2);
	\path (B2) + (\angB-90:3.2) coordinate (C2);
	\path (C2) + (\angB:2.1) coordinate (D2);
	\draw[fill=black] (A2) -- (B2) -- (C2) -- (D2) -- cycle;
	
	% Case
	\path (\M) + (\angB+180:2.1) coordinate (H1);
	\path (H1) + (\angB+90:8.1) coordinate (A1);
	\path (A1) + (\angB+180:7.1) coordinate (B1);
	\path (B1) + (\angB-90:16.2) coordinate (C1);
	\path (C1) + (\angB:7.1) coordinate (D1);	
	\draw[fill=gray] (A1) -- (B1) -- (C1) -- (D1) -- cycle;
	
	% Beam
	%\path (\M) + (\angB+90:\bwidth/2) coordinate (BI1);
	%\path (\M) + (\angB-90:\bwidth/2) coordinate (BI2);
	
	%\fill[color = red, opacity = \op] (BO1) -- (BO2) -- (BI1) -- (BI2) -- cycle;
}

\newcommand{\cameraT}[3]
{
	%#1 coordinates of camera
	%#2 coordinates of last optic	
	%#3 beam width
		
	\tikzmath{
	coordinate \M, \Next;
	\M = (#1);
	\Next = (#2);	
	\bwidth = #3;
	\x = \Nextx - \Mx;
	\y = \Nexty - \My;	
	if \x < 0 then {
	\h = 180;
	} else {
	\h = 0;
	};
	\angB = atan(\y/\x) + \h;
	\outx = \Mx;
	\outy = \My;
	};
		
	% Tubus
	\path (\M) + (\angB+90:1.6) coordinate (A2);
	\path (A2) + (\angB+180:2.1) coordinate (B2);
	\path (B2) + (\angB-90:3.2) coordinate (C2);
	\path (C2) + (\angB:2.1) coordinate (D2);
	\draw[fill=black] (A2) -- (B2) -- (C2) -- (D2) -- cycle;
	
	% Case
	\path (\M) + (\angB+180:2.1) coordinate (H1);
	\path (H1) + (\angB+90:4.1) coordinate (A1);
	\path (A1) + (\angB+180:10.1) coordinate (B1);
	\path (B1) + (\angB-90:8.2) coordinate (C1);
	\path (C1) + (\angB:10.1) coordinate (D1);	
	\draw[fill=gray!50] (A1) -- (B1) -- (C1) -- (D1) -- cycle;
	
	% Beam
	\path (\M) + (\angB+90:\bwidth/2) coordinate (BI1);
	\path (\M) + (\angB-90:\bwidth/2) coordinate (BI2);
	
	\fill[color = red, opacity = \op] (BO1) -- (BO2) -- (BI1) -- (BI2) -- cycle;
}